\documentclass[12pt,a4paper]{report}

% ========================
% PACOTES BÁSICOS
% ========================
\usepackage[utf8]{inputenc}
\usepackage[T1]{fontenc}
\usepackage[main=portuguese,english]{babel}
\usepackage{graphicx}
\usepackage{geometry}
\usepackage{enumitem}
\usepackage{titlesec}
\usepackage{amsmath,amsfonts}
\usepackage{setspace}
\usepackage{booktabs}
\usepackage{float}
\usepackage{caption}
\usepackage[acronym]{glossaries}
\usepackage{hyperref}

\geometry{margin=2.5cm}
\onehalfspacing
\hypersetup{colorlinks=true, linkcolor=black, citecolor=black, urlcolor=black}

\titleformat{\chapter}{\normalfont\huge\bfseries}{\thechapter.}{1em}{}

\makeglossaries
\renewcommand*{\glossaryname}{Glossário}
\renewcommand*{\acronymname}{Lista de Siglas}

% ---------------------------------------------------------
% Definições de acrônimos
% ---------------------------------------------------------

\newacronym{FPGA}{FPGA}{Field Programmable Gate Array}
\newacronym{LWIR}{LWIR}{Long-Wave Infrared}
\newacronym{RAM}{RAM}{Random Access Memory}
\newacronym{BRAM}{BRAM}{Block Random Access Memory}
\newacronym{DTR}{DTR}{Documento Técnico de Referência}
\newacronym{JPEG}{JPEG}{Joint Photographic Experts Group}
\newacronym{VHDL}{VHDL}{VHSIC Hardware Description Language}
\newacronym{RTL}{RTL}{Register Transfer Level}
\newacronym{CRC}{CRC}{Cyclic Redundancy Check}
\newacronym{TIR}{TIR}{\textit{Thermal Infrared}}
\newacronym{AXI4-Stream}{AXI4-Stream}{\textit{Advanced eXtensible Interface 4 Stream}}
\newacronym{LZW}{LZW}{\textit{Lempel–Ziv–Welch}}
\newacronym{DPCM}{DPCM}{\textit{Differential Pulse Code Modulation}}
\newacronym{RLE}{RLE}{\textit{Run-Length Encoding}}
\newacronym{FF}{FF}{\textit{Flip-Flop}}
\newacronym{FIFO}{FIFO}{\textit{First-In, First-Out}}
\newacronym{ABNT}{ABNT}{Associação Brasileira de Normas Técnicas}
\newacronym{ANS}{ANS}{Asymmetric Numeral Systems}
\newacronym{CR}{CR}{Compression Ratio}
\newacronym{HLS}{HLS}{High-Level Synthesis}
\newacronym{DSP}{DSP}{Digital Signal Processor}
\newacronym{ILA}{ILA}{Integrated Logic Analyzer}
\newacronym{FMAX}{FMAX}{Maximum Operating Frequency}
\newacronym{SoC}{SoC}{System-on-a-Chip}
\newacronym{GPU}{GPU}{Graphics Processing Unit}
\newacronym{DRS}{DRS}{Documento de Requisitos de Sistema}
\newacronym{HDL}{HDL}{Hardware Description Language}
\newacronym{VHSIC}{VHSIC}{Very High Speed Integrated Circuit}
\newacronym{CPU}{CPU}{Central Processing Unit}
\newacronym{LUT}{LUT}{Look-Up Table}

% ---------------------------------------------------------
% Definições de termos técnicos
% ---------------------------------------------------------
\newglossaryentry{compressao}{
    name={compressão},
    description={Processo de redução do volume de dados mantendo a integridade da informação}
}

\newglossaryentry{imagemtermica}{
    name={imagem térmica},
    description={Imagem obtida com base na radiação infravermelha emitida por corpos}
}

\newglossaryentry{lossless}{
    name={sem perdas (lossless)},
    description={Tipo de compressão que permite reconstrução exata dos dados originais}
}

\newglossaryentry{RAW}{
    name={RAW},
    description={Formato de imagem bruta, sem compressão ou processamento}
}

\newglossaryentry{II}{
    name={II},
    description={\textit{Initiation Interval} — intervalo de iniciação em pipelines, representando o número de ciclos de clock entre o início de duas iterações consecutivas. Um valor de II=1 indica paralelismo máximo}
}

\newglossaryentry{ChipScope}{
    name={ChipScope},
    description={Conjunto de ferramentas de depuração \textit{on-chip} da Xilinx que permite a visualização de sinais internos do FPGA em tempo real, essencial para validação de hardware.}
}

\newglossaryentry{Zynq7020}{
    name={Zynq7020},
    description={Modelo de System-on-Chip (SoC) da Xilinx que integra um processador ARM e lógica de FPGA programável, frequentemente usado como alvo para sistemas embarcados de processamento de imagem.}
}

\newglossaryentry{Vivado}{
    name={Vivado},
    description={Suíte de software de desenvolvimento da Xilinx, utilizada para síntese, implementação e verificação de designs de hardware para FPGAs e SoCs da empresa.}
}

\newglossaryentry{DSP48}{
    name={DSP48},
    description={Bloco de hardware dedicado encontrado em FPGAs da Xilinx, otimizado para realizar operações aritméticas rápidas como multiplicação e acumulação, essencial para processamento de sinais digitais e algoritmos de compressão como DPCM.}
}

\newglossaryentry{Huffman}{
    name={Huffman},
    description={Algoritmo de codificação de entropia sem perdas que utiliza um conjunto de códigos de comprimento variável. Baseia-se na frequência de ocorrência dos símbolos, atribuindo códigos curtos a símbolos mais frequentes e códigos longos a símbolos menos frequentes.}
}

\newglossaryentry{Verilog}{
    name={Verilog},
    description={Uma das linguagens de descrição de hardware (HDL) mais utilizadas, padronizada pela IEEE. É usada para modelar, simular e sintetizar circuitos eletrônicos digitais para implementação em FPGAs e ASICs.}
}

\newglossaryentry{PynqZ2}{
    name={PynqZ2},
    description={Placa de desenvolvimento baseada no \gls{SoC} Xilinx \gls{Zynq7020}, utilizada em aplicações de aprendizado de máquina e processamento de imagem.}
}

% ========================
% INÍCIO DO DOCUMENTO
% ========================
\begin{document}

\begin{titlepage}
    \centering
    {\Huge \textbf{Invent Vision}}\\[3cm]
    {\Huge \textbf{Documento de Especificação \\de Requisitos}}\\[0.5cm]
    {\large \textbf{Compressão Sem Perdas de Imagens Térmicas \\LWIR em FPGA}}\\[3cm]
    {\large Versão 1.0}\\[1cm]
    {\large Autor: Ubiratan da Silva Tavares}\\[0.2cm]
    {\large Data: \today}\\[5cm]
    \vfill
\end{titlepage}

\tableofcontents
\newpage

% ========================
% CAPÍTULO 1 – INTRODUÇÃO
% ========================
\chapter{Introdução}

\section{Objetivo}
Este documento descreve os requisitos funcionais e não funcionais do sistema de compressão sem perdas de imagens térmicas LWIR implementado em FPGA, com o propósito de garantir rastreabilidade, desempenho e conformidade técnica.

\section{Escopo}
O sistema abrange a compressão e descompressão de dados de imagem LWIR de 16 bits, em tempo real, empregando algoritmos de compressão sem perdas otimizados para hardware reconfigurável.

\section{Convenções, Termos e Abreviações}
Os requisitos são identificados no formato \texttt{[RFxxx]} e \texttt{[RNFxxx]}, onde RF indica requisito funcional e RNF requisito não funcional.

\section{Prioridades dos Requisitos}
\begin{itemize}
\item \textbf{Essencial:} requisito indispensável ao funcionamento.
\item \textbf{Importante:} necessário, mas não crítico.
\item \textbf{Desejável:} opcional ou futuro.
\end{itemize}

\section{Referências}
Liste documentos de apoio, artigos e manuais do FPGA alvo.

% ========================
% CAPÍTULO 2 – DESCRIÇÃO GERAL DO SISTEMA
% ========================
\chapter{Descrição Geral do Sistema}

\section{Visão Geral}
O sistema realiza compressão sem perdas em fluxo contínuo (\textit{streaming}) de imagens LWIR de 16 bits, com arquitetura paralela e baixa latência.

\section{Usuários e Ambiente de Operação}
\begin{itemize}
\item \textbf{Engenheiros de Hardware:} responsáveis pela síntese e integração no FPGA.
\item \textbf{Pesquisadores em Visão Computacional:} validam os resultados da compressão.
\end{itemize}

\section{Dependências e Interfaces}
O sistema integra-se a módulos de captura de imagem via barramento AXI4-Stream e exporta dados comprimidos para memória ou transmissor externo.

% ========================
% CAPÍTULO 3 – REQUISITOS FUNCIONAIS
% ========================
\chapter{Requisitos Funcionais}

\section{Requisitos}

\subsection{[RF001] Compressão Sem Perdas LWIR}
O sistema deve realizar compressão sem perdas de imagens LWIR de 16 bits em tempo real.

\subsection{[RF002] Interface de Entrada e Saída}
Deve aceitar dados de imagem via interface AXI4-Stream e gerar saída comprimida compatível com o pipeline de processamento.

\subsection{[RF003] Algoritmo LZW em Hardware}
O sistema deve implementar o algoritmo LZW em arquitetura paralela com suporte a dicionário dinâmico.

\subsection{[RF004] Configuração e Modos de Operação}
Deve permitir configuração entre modos de compressão DPCM-RLE (baseline) e LZW otimizado.

\subsection{[RF005] Monitoramento de Desempenho}
O sistema deve calcular e disponibilizar métricas de taxa de compressão (CR) e throughput.

% ========================
% CAPÍTULO 4 – REQUISITOS NÃO FUNCIONAIS
% ========================
\chapter{Requisitos Não Funcionais}

\section{Desempenho}
\begin{itemize}
\item [RNF001] O sistema deve atingir uma Taxa de Compressão mínima (CR) definida após caracterização dos dados.
\item [RNF002] A latência total de processamento deve ser inferior a 1 ms por frame.
\end{itemize}

\section{Arquitetura e Design de Hardware}
\begin{itemize}
\item [RNF003] O algoritmo LZW deve empregar BRAMs de porta dupla para leitura e escrita simultânea.
\item [RNF004] A arquitetura deve ser totalmente pipeline (\textit{fully pipelined}) para alta taxa de transferência.
\end{itemize}

\section{Recursos de Hardware}
\begin{itemize}
\item [RNF005] O design deve otimizar o uso de LUTs, FFs, BRAMs e DSPs conforme a FPGA alvo.
\end{itemize}

\section{Confiabilidade e Segurança}
\begin{itemize}
\item [RNF006] A compressão deve ser determinística e reprodutível.
\item [RNF007] O sistema deve prevenir corrupção de dados e estouros de buffer.
\end{itemize}

\section{Usabilidade e Portabilidade}
\begin{itemize}
\item [RNF008] O design deve ser portável entre famílias de FPGA.
\item [RNF009] Deve integrar-se facilmente a sistemas embarcados de captura e processamento.
\end{itemize}

% ========================
% CAPÍTULO 5 – AMBIENTE DE DESENVOLVIMENTO
% ========================
\chapter{Requisitos de Ambiente e Restrições Finais}

\section{Requisitos de Desempenho Fechados}

\subsection*{[RNF010] Taxa de Compressão Mínima (CR Fechado)}
O acelerador de compressão deve alcançar uma Taxa de Compressão (CR) na faixa de \textbf{1.5:1 a 2.5:1} \textit{lossless} sobre os dados \gls{TIR} de 16 bits.

\subsection*{[RNF011] Latência de Processamento (Requisito Fechado)}
A latência total de processamento deve ser inferior a \textbf{1 ms} por \textit{frame} de imagem.

\section{Ambiente de Desenvolvimento (Toolchain)}

\subsection*{[RNF012] Ferramenta de Síntese/Implementação}
O código \gls{RTL} deve ser sintetizável na suíte \textbf{Xilinx Vivado} (e.g., Vivado 2023.2).

\subsection*{[RNF013] Verificação de Hardware}
O design deve ser compatível com ferramentas de análise de \textit{timing} e depuração \textit{on-chip} (como \gls{ILA} ou \gls{ChipScope}) para a validação da Corretude.

\section{Pontos Abertos Finais (A.4)}
\label{aberto:fechamento}

O DRS v0.1 está finalizado, mas exige a resolução formal dos seguintes pontos na fase de implementação:
\begin{itemize}
    \item \textbf{Orçamento de Recursos (FPGA Alvo):} Definir e fixar o \textbf{limite máximo de utilização} de \gls{LUT}s, \gls{FF}s, \gls{BRAM}s e \gls{DSP}s no \gls{FPGA} alvo (e.g., Xilinx \gls{Zynq7020}) para o subsistema de compressão (RFN-5).
    \item \textbf{Datasets de Validação:} Finalizar a aquisição e caracterização dos \textbf{Datasets de Imagens \gls{TIR} de 16 bits} para realizar os testes de CR e Corretude (RFN-6.1 e RFN-6.2).
\end{itemize}


% ========================
% CAPÍTULO 6 – GLOSSÁRIO
% ========================
\printglossaries

\end{document}