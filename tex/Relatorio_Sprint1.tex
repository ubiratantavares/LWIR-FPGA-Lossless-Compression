\documentclass[12pt,a4paper]{article}

% Pacotes básicos
\usepackage[utf8]{inputenc}
\usepackage[T1]{fontenc}
\usepackage[brazil]{babel}
\usepackage{graphicx}
\usepackage{geometry}
\usepackage{booktabs}
\usepackage{float}
\usepackage{hyperref}
\usepackage{listings}
\usepackage{xcolor}

% Configuração de margens
\geometry{margin=2.5cm}

% Configuração de código
\definecolor{codegreen}{rgb}{0,0.6,0}
\definecolor{codegray}{rgb}{0.5,0.5,0.5}
\definecolor{codepurple}{rgb}{0.58,0,0.82}
\definecolor{backcolour}{rgb}{0.95,0.95,0.92}

\lstdefinestyle{verilogstyle}{
    backgroundcolor=\color{backcolour},   
    commentstyle=\color{codegreen},
    keywordstyle=\color{magenta},
    numberstyle=\tiny\color{codegray},
    stringstyle=\color{codepurple},
    basicstyle=\ttfamily\footnotesize,
    breakatwhitespace=false,         
    breaklines=true,                 
    captionpos=b,                    
    keepspaces=true,                 
    numbers=left,                    
    numbersep=5pt,                  
    showspaces=false,                
    showstringspaces=false,
    showtabs=false,                  
    tabsize=2
}

\lstset{style=verilogstyle}

% Título e Autor
\title{\textbf{Relatório Parcial de Execução da Sprint 1}\\ \large Implementação do Baseline DPCM-RLE em Verilog}
\author{Invent Vision - Projeto LWIR FPGA Lossless Compression}
\date{07 de Janeiro de 2026}

\begin{document}

\maketitle

\section{Introdução}
A \textbf{Sprint 1} tem como objetivo implementar e validar a primeira versão funcional do sistema de compressão em hardware. Este relatório parcial documenta as atividades de codificação HDL (Passo 1) e a preparação do ambiente de verificação (Passo 2.1), conforme definido no planejamento.

\section{Implementação HDL (Passo 1)}

Foi desenvolvido o pipeline de compressão \textit{Baseline} utilizando a linguagem Verilog 2005. A arquitetura é composta por dois estágios principais operando em fluxo contínuo (\textit{streaming}).

\subsection{Preditor DPCM Refinado}
O módulo \texttt{DPCM\_Predictor.v} foi atualizado para garantir robustez na integração.
\begin{itemize}
    \item \textbf{Funcionalidade:} Implementa a predição \textit{Left Neighbor} ($R_i = P_i - P_{i-1}$).
    \item \textbf{Interface:} Adicionado sinal de \texttt{valid\_in}/\texttt{valid\_out} para propagar o controle de fluxo.
    \item \textbf{Dados:} Entrada de 16 bits (unsigned) e saída de 17 bits (signed) para evitar overflow aritmético.
\end{itemize}

\subsection{Codificador RLE (Zero-Run)}
O módulo \texttt{RLE\_Encoder.v} foi implementado para explorar a redundância espacial dos resíduos.
\begin{itemize}
    \item \textbf{Estratégia:} O codificador detecta sequências de zeros (comuns em áreas térmicas uniformes) e as agrupa.
    \item \textbf{Formato de Saída:} Pacotes de 32 bits contendo:
    \begin{itemize}
        \item \textbf{Bits [31:17]:} Contador de Zeros (15 bits).
        \item \textbf{Bits [16:0]:} Literal Não-Zero (17 bits).
    \end{itemize}
    \item \textbf{Eficiência:} Esta abordagem permite uma taxa de compressão 1:1 no pior caso (sem zeros) e alta compressão em regiões planas, sem introduzir \textit{stalls} no pipeline (throughput de 1 pixel/ciclo).
\end{itemize}

\subsection{Integração (Top-Level)}
O módulo \texttt{LWIR\_Lossless\_Compression.v} instancia e conecta os blocos:
\begin{lstlisting}[language=Verilog, caption=Estrutura do Top-Level]
    DPCM_Predictor inst_dpcm ( ... );
    RLE_Encoder inst_rle ( ... );
\end{lstlisting}
O fluxo de dados é totalmente síncrono e controlado pelos sinais de validade.

\section{Ambiente de Validação (Passo 2)}

Para garantir a corretude \textit{bit-perfect} do hardware, foi estabelecido um ambiente de verificação automatizado.

\subsection{Testbench RTL}
O arquivo \texttt{tb\_LWIR\_Lossless\_Compression.v} foi criado para simular o circuito.
\begin{itemize}
    \item \textbf{Estímulo:} Lê pixels de um arquivo \texttt{input.hex}.
    \item \textbf{Clock:} Gera um clock de 100 MHz (período de 10ns).
    \item \textbf{Captura:} Salva a saída válida em \texttt{output.hex}.
\end{itemize}

\subsection{Geração de Vetores}
Um script Python (\texttt{generate\_vectors.py}) foi desenvolvido para converter imagens reais do dataset FLIR (TIFF 16-bit) para o formato hexadecimal esperado pelo testbench. O arquivo \texttt{input.hex} já foi gerado com sucesso a partir do primeiro frame do dataset.

\section{Próximos Passos}
As atividades de codificação e preparação do ambiente de teste foram concluídas. Os próximos passos para finalizar a Sprint 1 são:
\begin{enumerate}
    \item Executar a simulação funcional (ModelSim/Icarus) para gerar o \texttt{output.hex}.
    \item Validar a saída utilizando o script \texttt{validate\_output.py}.
    \item Realizar a síntese no Quartus II para obter métricas de área e frequência.
\end{enumerate}

\end{document}
