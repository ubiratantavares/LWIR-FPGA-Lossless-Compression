% =========================================================
% Documento Técnico de Referência (DTR)
% Modelo LaTeX com glossário, acrônimos e estrutura modular
% =========================================================

\documentclass[12pt,a4paper]{report}

% ---------------------------------------------------------
% Codificação, idioma e formatação
% ---------------------------------------------------------
\usepackage[utf8]{inputenc}
\usepackage[T1]{fontenc}
\usepackage[brazil]{babel}
\usepackage{lmodern}
\usepackage{setspace}
\usepackage{geometry}
\usepackage{graphicx}
\usepackage{booktabs}
\usepackage{hyperref}
\usepackage{amsmath, amssymb}
\usepackage{xcolor}

\geometry{a4paper, margin=2.5cm}
\onehalfspacing
\setlength{\parindent}{1.25cm}
\setlength{\parskip}{0.2cm}

% ---------------------------------------------------------
% Configuração de links e sumário
% ---------------------------------------------------------
\hypersetup{
    colorlinks=false,
    linkcolor=black,
    urlcolor=black,
    citecolor=black,
    pdfauthor={Ubiratan da Silva Tavares},
    pdftitle={Documento Técnico de Referência}
}

% ---------------------------------------------------------
% Pacote de glossários e acrônimos
% ---------------------------------------------------------
\usepackage[acronym, toc, nonumberlist]{glossaries}
\makeglossaries

% ---------------------------------------------------------
% Definições de acrônimos e termos técnicos
% ---------------------------------------------------------

% ---------------------------------------------------------
% Definições de acrônimos
% ---------------------------------------------------------

\newacronym{FPGA}{FPGA}{Field Programmable Gate Array}
\newacronym{LWIR}{LWIR}{Long-Wave Infrared}
\newacronym{RAM}{RAM}{Random Access Memory}
\newacronym{BRAM}{BRAM}{Block Random Access Memory}
\newacronym{DTR}{DTR}{Documento Técnico de Referência}
\newacronym{JPEG}{JPEG}{Joint Photographic Experts Group}
\newacronym{VHDL}{VHDL}{VHSIC Hardware Description Language}
\newacronym{RTL}{RTL}{Register Transfer Level}
\newacronym{CRC}{CRC}{Cyclic Redundancy Check}
\newacronym{TIR}{TIR}{\textit{Thermal Infrared}}
\newacronym{AXI4-Stream}{AXI4-Stream}{\textit{Advanced eXtensible Interface 4 Stream}}
\newacronym{LZW}{LZW}{\textit{Lempel–Ziv–Welch}}
\newacronym{DPCM}{DPCM}{\textit{Differential Pulse Code Modulation}}
\newacronym{RLE}{RLE}{\textit{Run-Length Encoding}}
\newacronym{FF}{FF}{\textit{Flip-Flop}}
\newacronym{FIFO}{FIFO}{\textit{First-In, First-Out}}
\newacronym{ABNT}{ABNT}{Associação Brasileira de Normas Técnicas}
\newacronym{ANS}{ANS}{Asymmetric Numeral Systems}
\newacronym{CR}{CR}{Compression Ratio}
\newacronym{HLS}{HLS}{High-Level Synthesis}
\newacronym{DSP}{DSP}{Digital Signal Processor}
\newacronym{ILA}{ILA}{Integrated Logic Analyzer}
\newacronym{FMAX}{FMAX}{Maximum Operating Frequency}
\newacronym{SoC}{SoC}{System-on-a-Chip}
\newacronym{GPU}{GPU}{Graphics Processing Unit}
\newacronym{DRS}{DRS}{Documento de Requisitos de Sistema}
\newacronym{HDL}{HDL}{Hardware Description Language}
\newacronym{VHSIC}{VHSIC}{Very High Speed Integrated Circuit}
\newacronym{CPU}{CPU}{Central Processing Unit}
\newacronym{LUT}{LUT}{Look-Up Table}

% ---------------------------------------------------------
% Definições de termos técnicos
% ---------------------------------------------------------
\newglossaryentry{compressao}{
    name={compressão},
    description={Processo de redução do volume de dados mantendo a integridade da informação}
}

\newglossaryentry{imagemtermica}{
    name={imagem térmica},
    description={Imagem obtida com base na radiação infravermelha emitida por corpos}
}

\newglossaryentry{lossless}{
    name={sem perdas (lossless)},
    description={Tipo de compressão que permite reconstrução exata dos dados originais}
}

\newglossaryentry{RAW}{
    name={RAW},
    description={Formato de imagem bruta, sem compressão ou processamento}
}

\newglossaryentry{II}{
    name={II},
    description={\textit{Initiation Interval} — intervalo de iniciação em pipelines, representando o número de ciclos de clock entre o início de duas iterações consecutivas. Um valor de II=1 indica paralelismo máximo}
}

\newglossaryentry{ChipScope}{
    name={ChipScope},
    description={Conjunto de ferramentas de depuração \textit{on-chip} da Xilinx que permite a visualização de sinais internos do FPGA em tempo real, essencial para validação de hardware.}
}

\newglossaryentry{Zynq7020}{
    name={Zynq7020},
    description={Modelo de System-on-Chip (SoC) da Xilinx que integra um processador ARM e lógica de FPGA programável, frequentemente usado como alvo para sistemas embarcados de processamento de imagem.}
}

\newglossaryentry{Vivado}{
    name={Vivado},
    description={Suíte de software de desenvolvimento da Xilinx, utilizada para síntese, implementação e verificação de designs de hardware para FPGAs e SoCs da empresa.}
}

\newglossaryentry{DSP48}{
    name={DSP48},
    description={Bloco de hardware dedicado encontrado em FPGAs da Xilinx, otimizado para realizar operações aritméticas rápidas como multiplicação e acumulação, essencial para processamento de sinais digitais e algoritmos de compressão como DPCM.}
}

\newglossaryentry{Huffman}{
    name={Huffman},
    description={Algoritmo de codificação de entropia sem perdas que utiliza um conjunto de códigos de comprimento variável. Baseia-se na frequência de ocorrência dos símbolos, atribuindo códigos curtos a símbolos mais frequentes e códigos longos a símbolos menos frequentes.}
}

\newglossaryentry{Verilog}{
    name={Verilog},
    description={Uma das linguagens de descrição de hardware (HDL) mais utilizadas, padronizada pela IEEE. É usada para modelar, simular e sintetizar circuitos eletrônicos digitais para implementação em FPGAs e ASICs.}
}

\newglossaryentry{PynqZ2}{
    name={PynqZ2},
    description={Placa de desenvolvimento baseada no \gls{SoC} Xilinx \gls{Zynq7020}, utilizada em aplicações de aprendizado de máquina e processamento de imagem.}
}

% ---------------------------------------------------------
% Início do documento
% ---------------------------------------------------------
\begin{document}

% ---------------------------------------------------------
% Capa
% ---------------------------------------------------------
\begin{titlepage}
    \centering
    {\Huge \textbf{Invent Vision}}\\[3cm]
    {\Huge \textbf{Documento Técnico de Referência}}\\[0.5cm]
    {\Large \textbf{Compressão Sem Perdas de Imagens Térmicas \\LWIR em FPGA}}\\[3cm]
    {\large Autor: Ubiratan da Silva Tavares}\\[0.2cm]
    {\large Data: \today}\\[5cm]
    \vfill
\end{titlepage}

% ---------------------------------------------------------
% Sumário
% ---------------------------------------------------------
\tableofcontents
\newpage

% ---------------------------------------------------------
% Capítulo 1
% ---------------------------------------------------------
\chapter{Introdução}

\section{Contextualização}

O avanço das tecnologias de imagem térmica na faixa do \textit{Long Wave Infrared} (LWIR), entre 8--14~µm, tem possibilitado aplicações em áreas como vigilância, sensoriamento remoto, automação industrial e diagnóstico preditivo. No entanto, essas aplicações enfrentam desafios significativos relacionados ao grande volume de dados gerados pelos sensores térmicos, especialmente em sistemas embarcados de processamento em tempo real.

Em plataformas baseadas em \gls{FPGA}, a necessidade de transmitir, armazenar e processar imagens térmicas sem perdas de informação torna o projeto de sistemas de compressão um componente crítico. Soluções tradicionais de compressão \textit{lossy}, como JPEG, não são adequadas para cenários que exigem integridade absoluta dos dados radiométricos, uma vez que qualquer perda de precisão compromete análises quantitativas e calibrações térmicas.

\section{Motivação e Desafios}

A implementação de um sistema de compressão \textit{lossless} eficiente para imagens LWIR em FPGA requer o equilíbrio entre desempenho, uso de recursos lógicos e complexidade arquitetural. Além disso, há a necessidade de garantir latência previsível e throughput sustentado, mesmo em dispositivos de médio porte, como o \gls{Zynq7020}.

Outro fator motivador é a carência de soluções abertas e documentadas que conciliem compressão sem perdas, eficiência em \gls{DSP}, e integração direta com fluxos de vídeo térmico em plataformas \gls{SoC}. A literatura existente aborda predominantemente algoritmos de compressão voltados a \textit{software} ou a implementações em \gls{GPU}, que não atendem aos requisitos de sistemas embarcados determinísticos.

\section{Objetivo Geral}

O objetivo principal deste projeto é projetar e implementar uma arquitetura de compressão sem perdas de imagens térmicas LWIR em FPGA, priorizando:
\begin{itemize}
    \item Eficiência no uso de recursos lógicos e de memória;
    \item Alto \textit{throughput} com latência mínima;
    \item Preservação completa dos dados radiométricos originais;
    \item Facilidade de integração com sistemas de captura e transmissão existentes.
\end{itemize}

\section{Objetivos Específicos}

Para alcançar o objetivo geral, são definidos os seguintes objetivos específicos:
\begin{enumerate}
    \item Selecionar e adaptar algoritmos de compressão sem perdas adequados ao domínio de imagens térmicas;
    \item Modelar e validar a arquitetura proposta em ambiente de simulação;
    \item Implementar e otimizar o projeto em hardware, utilizando ferramentas de síntese de alto nível (\gls{HLS});
    \item Avaliar o desempenho em termos de \gls{CR}, \gls{FMAX}, \gls{II}, utilização de \gls{DSP} e consumo de \gls{BRAM};
    \item Documentar os requisitos funcionais e não funcionais, assegurando rastreabilidade com os resultados obtidos.
\end{enumerate}

\section{Justificativa}

A compressão sem perdas em FPGA para dados térmicos representa um campo de relevância crescente em sistemas embarcados de visão computacional. A ausência de soluções padronizadas e a necessidade de desempenho determinístico motivam o desenvolvimento de uma implementação modular, configurável e reprodutível, que sirva como referência acadêmica e base tecnológica para aplicações industriais.

Além disso, o domínio LWIR apresenta peculiaridades como alta correlação espacial entre pixels e ruído térmico não gaussiano, o que requer estratégias de predição e codificação de entropia específicas. Assim, a proposta deste projeto busca contribuir para o avanço do estado da arte em compressão \textit{on-chip}, oferecendo um framework escalável que possa ser reutilizado em diferentes contextos de aquisição térmica.

\section{Estrutura do Documento}

Este documento está organizado da seguinte forma:
\begin{itemize}
    \item O \textbf{Capítulo 1} apresenta a introdução, contextualização, motivação, objetivos e justificativas do projeto.
    \item O \textbf{Capítulo 2} discute os fundamentos teóricos da compressão sem perdas, abordando modelos preditivos e técnicas de codificação de entropia.
    \item O \textbf{Capítulo 3} descreve a arquitetura proposta, o ambiente de implementação e as decisões técnicas do \gls{DTR}.
    \item O \textbf{Capítulo 4} detalha os requisitos funcionais e não funcionais formalizados no \gls{DRS}.
    \item O \textbf{Capítulo 5} apresenta os resultados e discussões experimentais.
    \item O \textbf{Capítulo 6} contém as conclusões e perspectivas de continuidade.
\end{itemize}


% ---------------------------------------------------------
% Capítulo 2
% ---------------------------------------------------------
\chapter{Fundamentação Teórica}

\section{Visão Geral sobre Imagens Térmicas LWIR}

As imagens térmicas capturadas na faixa espectral \textit{Long-Wave Infrared} (LWIR), entre 8 e 14~µm, representam o mapa de radiação emitida por objetos, independentemente de iluminação visível. Diferentemente das imagens RGB, que codificam cor e intensidade luminosa, as imagens LWIR carregam informações de temperatura superficial, sendo essenciais para aplicações em sensoriamento remoto, inspeção industrial, vigilância e diagnóstico térmico.

Os sensores LWIR, geralmente baseados em microbolômetros, produzem dados com alta correlação espacial e dinâmica radiométrica de 12 a 16 bits por pixel. Essa densidade informacional resulta em fluxos de dados volumosos, exigindo técnicas de compressão que preservem a integridade radiométrica — especialmente quando a análise quantitativa é crítica.

\section{Compressão de Imagens}

A compressão de imagens tem como objetivo reduzir a redundância dos dados, minimizando o volume de armazenamento e transmissão sem comprometer a utilidade da informação. Ela se classifica em dois tipos principais:
\begin{itemize}
    \item \textbf{Compressão com perdas (\textit{lossy})}, na qual parte da informação original é descartada de forma irreversível, visando maior taxa de compressão;
    \item \textbf{Compressão sem perdas (\textit{lossless})}, em que o dado original pode ser totalmente reconstruído, preservando a precisão dos valores.
\end{itemize}

Para imagens térmicas LWIR, apenas o segundo tipo é aceitável, já que qualquer distorção introduzida pode alterar medições radiométricas ou comprometer calibrações térmicas.

\section{Modelos de Predição}

A compressão sem perdas baseia-se, em grande parte, na eliminação de redundâncias estatísticas entre pixels adjacentes. Modelos de predição estimam o valor de um pixel a partir de seus vizinhos e armazenam apenas o erro de predição (resíduo), que tende a apresentar menor entropia.

O modelo \textbf{\gls{DPCM}} (Differential Pulse Code Modulation) é amplamente utilizado nesse contexto. Ele calcula a diferença entre o valor real e o valor previsto do pixel:
\[
e(i,j) = I(i,j) - \hat{I}(i,j)
\]
onde $I(i,j)$ representa o valor do pixel atual e $\hat{I}(i,j)$ é o valor predito.  
Esses resíduos são então processados por um codificador de entropia, reduzindo o número médio de bits necessários para sua representação.

\section{Codificação de Entropia}

A etapa de codificação de entropia tem a função de representar os símbolos residuais com base em suas probabilidades de ocorrência. Entre os métodos mais comuns estão:
\begin{itemize}
    \item \textbf{\gls{Huffman}} — utiliza árvores de decisão para associar códigos de comprimento variável a símbolos de maior frequência;
    \item \textbf{\gls{LZW}} — emprega dicionários dinâmicos para substituição de sequências repetitivas;
    \item \textbf{\gls{RLE}} — substitui repetições consecutivas pelo par símbolo-contagem;
    \item \textbf{\gls{ANS}} (\textit{Asymmetric Numeral Systems}) — oferece eficiência próxima à codificação aritmética com menor complexidade computacional.
\end{itemize}

A escolha do método de entropia impacta diretamente a taxa de compressão (\gls{CR}), a latência e o uso de recursos lógicos, sendo necessária a análise de custo-benefício conforme a arquitetura do \gls{FPGA}.

\section{Arquiteturas FPGA e Síntese de Alto Nível}

Dispositivos \gls{FPGA} (\textit{Field Programmable Gate Array}) são compostos por blocos lógicos programáveis (\gls{LUT}), elementos de memória (\gls{BRAM}), e unidades dedicadas de processamento aritmético (\gls{DSP}). Essa flexibilidade os torna ideais para aplicações de processamento paralelo e pipelines de dados.

A utilização de ferramentas de \gls{HLS} (\textit{High-Level Synthesis}) permite descrever o comportamento da arquitetura em linguagens de alto nível, como C/C++, e gerar automaticamente descrições em \gls{HDL}. Isso reduz o tempo de desenvolvimento e facilita a exploração de parâmetros como:
\begin{itemize}
    \item \textbf{\gls{II} (Initiation Interval)} — número de ciclos de clock entre o início de duas iterações consecutivas no pipeline;
    \item \textbf{\gls{FMAX}} — frequência máxima de operação atingível após síntese e implementação;
    \item \textbf{\gls{DSP} Utilization} — quantidade de blocos dedicados utilizados nas operações aritméticas.
\end{itemize}

\section{Métricas de Avaliação}

A eficiência de um sistema de compressão implementado em hardware é medida por métricas complementares:
\begin{itemize}
    \item \textbf{Taxa de Compressão (\gls{CR})} — relação entre o tamanho dos dados originais e o tamanho dos dados comprimidos;
    \item \textbf{Throughput} — quantidade de dados processados por unidade de tempo;
    \item \textbf{Latência} — tempo entre a entrada do primeiro pixel e a saída do primeiro dado comprimido;
    \item \textbf{Uso de Recursos} — fração de LUTs, BRAMs e DSPs ocupados;
    \item \textbf{Energia por Bit Processado} — métrica de eficiência energética relevante para aplicações embarcadas.
\end{itemize}

Essas métricas permitem avaliar o compromisso entre desempenho e custo de implementação, orientando otimizações sucessivas durante o processo de síntese.

\section{Considerações Finais}

A fundamentação teórica apresentada fornece o arcabouço necessário para compreender os princípios de compressão sem perdas, as técnicas de predição e codificação, bem como os aspectos arquiteturais envolvidos na implementação em \gls{FPGA}.  
Os conceitos aqui descritos sustentam o desenvolvimento da arquitetura proposta e a análise dos resultados obtidos nos capítulos seguintes.

% ---------------------------------------------------------
% Capítulo 3
% ---------------------------------------------------------
\chapter{Arquitetura e Implementação}
\label{chap:arquitetura}

\section{Visão Geral da Arquitetura Proposta}

A arquitetura proposta tem como objetivo realizar a compressão sem perdas de imagens térmicas LWIR em \gls{FPGA}, preservando integralmente a radiometria dos dados e otimizando o uso de recursos lógicos. O sistema é concebido de forma modular, com blocos independentes conectados em um fluxo de dados contínuo (\textit{streaming pipeline}).

O projeto prioriza o equilíbrio entre desempenho e eficiência, buscando atingir \gls{II} = 1 e operação em frequência máxima próxima à \gls{FMAX} teórica do dispositivo, sem ultrapassar 50\% de utilização de \gls{LUT}s e \gls{BRAM}s.  

\section{Estrutura Funcional}

Os principais módulos são:

\begin{itemize}
    \item \textbf{Módulo de Aquisição e Bufferização:} Responsável por receber os dados de entrada (pixels de 12 a 16 bits) e armazená-los em \gls{FIFO} para alimentação ordenada do pipeline.
    \item \textbf{Módulo de Predição (DPCM):} Implementa o modelo diferencial pixel-a-pixel, calculando o erro $e(i,j) = I(i,j) - \hat{I}(i,j)$. Este bloco é altamente paralelizável e utiliza \gls{DSP}s dedicados.
    \item \textbf{Módulo de Codificação RLE:} Aplica \textit{Run-Length Encoding} para reduzir redundâncias de sequência, usando contador de 3 bits.
    \item \textbf{Módulo de Codificação LZW:} Opera como codificador de dicionário dinâmico, aproveitando \gls{BRAM}s de porta dupla para implementação eficiente da tabela \textit{hash}.
    \item \textbf{Controlador de Fluxo e Interface:} Coordena a comunicação entre módulos e realiza o controle de sincronismo, garantindo operação contínua sem gargalos.
\end{itemize}

Cada módulo é sintetizável de forma independente, permitindo a substituição incremental durante o ciclo de desenvolvimento e testes.

\section{Modelo de Pipeline e Paralelismo}

A arquitetura é projetada em pipeline com quatro estágios principais:
\begin{enumerate}
    \item \textbf{Leitura e Previsão (Stage 1)} — leitura dos pixels e predição \gls{DPCM};
    \item \textbf{Cálculo de Erro e Codificação RLE (Stage 2)};
    \item \textbf{Codificação LZW (Stage 3)};
    \item \textbf{Empacotamento e Saída (Stage 4)}.
\end{enumerate}

A configuração busca atingir \gls{II}=1, ou seja, o início de uma nova operação a cada ciclo de clock. Essa abordagem maximiza o \textit{throughput} mantendo a latência sob controle, uma vez que o pipeline totaliza quatro ciclos até a produção do primeiro resultado.

\section{Uso de Recursos e Otimizações}

\subsection{Uso de Blocos Dedicados}

Para maximizar a eficiência, são utilizados blocos dedicados do \gls{FPGA}:
\begin{itemize}
    \item \gls{DSP48} — para operações aritméticas do preditor e cálculo de diferenças.
    \item \gls{BRAM}s — para armazenar o dicionário do \gls{LZW} e as filas intermediárias do pipeline.
    \item \gls{FF}s — utilizados em estratégias de \textit{retiming} e \textit{pipelining} para manter a estabilidade temporal.
\end{itemize}

\subsection{Otimizações de Síntese}

A etapa de \gls{HLS} é configurada com diretivas para otimizar latência e throughput:
\begin{itemize}
    \item \texttt{\#pragma HLS PIPELINE II=1} — garante início de operação a cada ciclo.
    \item \texttt{\#pragma HLS UNROLL factor=4} — explora paralelismo interno em loops críticos.
    \item \texttt{\#pragma HLS RESOURCE variable=hash core=RAM\_2P\_BRAM} — implementa a tabela hash do \gls{LZW} em BRAM de porta dupla.
\end{itemize}

\section{Ferramentas e Ambiente de Implementação}

O desenvolvimento utiliza a suíte \textbf{Xilinx Vivado} para síntese, simulação e geração do bitstream:
\begin{itemize}
    \item \textbf{Síntese e Implementação:} Vivado 2023.2, com suporte ao fluxo de \gls{HLS} e \gls{RTL};
    \item \textbf{Simulação:} Ambiente de \textit{testbench} com arquivos de entrada \textit{.dat} e \textit{.bin};
    \item \textbf{Plataforma de Referência:} \gls{Zynq7020}, com placa \gls{PynqZ2}, operando em 100 MHz;
    \item \textbf{Linguagem de Descrição:} C/C++ para \gls{HLS}, com integração em \gls{VHDL}/\gls{Verilog} para verificação estrutural.
\end{itemize}

A prototipagem inicial foca na implementação da lógica \textbf{Preditora DPCM + Codificador RLE}, considerada a versão \textit{Baseline Rápida} do sistema. Essa escolha reduz a complexidade lógica e permite validar a corretude funcional (\textit{lossless}) antes da integração do módulo LZW.

\section{Avaliação e Métricas Esperadas}

Os parâmetros de desempenho a serem avaliados incluem:
\begin{itemize}
    \item \textbf{Taxa de Compressão (\gls{CR}) mínima:} 1.8:1;
    \item \textbf{Throughput:} superior a 1 pixel/ciclo em 100 MHz;
    \item \textbf{Latência:} inferior a 4 ciclos entre entrada e saída do primeiro dado;
    \item \textbf{Utilização de Recursos:} abaixo de 50\% do total do \gls{Zynq7020}.
\end{itemize}

Esses indicadores refletem o alinhamento entre os requisitos funcionais e não funcionais definidos no DRS e as decisões de design documentadas no DTR.

\section{Conclusão Parcial do Sprint 1}

O encerramento da fase de arquitetura confirma a viabilidade técnica da compressão sem perdas de imagens LWIR em \gls{FPGA}.  
As análises e decisões de projeto estabelecem uma base sólida para a etapa de implementação prática, com o pipeline modular pronto para síntese e verificação no ambiente Vivado.

O foco do \textbf{Sprint 2} será a validação funcional do pipeline DPCM-RLE, medição experimental de throughput, latência e uso de recursos, seguida pela integração progressiva do módulo LZW e ajustes de performance.

\section{Decisão Final de Candidatos Primários}
\label{sec:candidatos}

Com base na Matriz de \textit{Trade-offs} e nos requisitos de alto \textit{throughput} e preservação radiométrica, os candidatos primários para a implementação em \gls{FPGA} são:

\begin{itemize}
    \item \textbf{Candidato de Alto Desempenho (Dicionário):} O algoritmo \textbf{\gls{LZW}} é selecionado devido à sua alta capacidade de paralelismo em arquiteturas de \textit{hardware} (até $23.51\times$ de \textit{speedup} com 24 PEs em paralelo) e o uso eficiente de \gls{BRAM}s de porta dupla para o dicionário \textit{hash}.
    \item \textbf{Candidato de Alta Eficiência (Predição):} Arquiteturas baseadas em \textbf{LOCO-I} (\textit{Lossless and Near-Lossless Compression}), que utilizam um \textbf{Preditor Fixo} e um \textbf{Codificador de Entropia} (como \gls{ANS} ou DPCM-RLE), são cruciais para dados de \textit{high bit depth} de imagens \gls{TIR}. O design DPCM/LOCO-I otimizado para compressão \textit{lossless} pode atingir II de 1, com um \textit{pipeline} de 4 estágios.
\end{itemize}

A estratégia de design (Decisão Protótipo) focará inicialmente na arquitetura \gls{LZW} otimizada para dicionário e no DPCM-RLE simples (utilizando contador de 3 bits, conforme referência), dada a sua baixa complexidade de \textit{hardware} e rapidez de prototipagem.

\section{Fechamento de Requisitos e Orçamento de Recursos}

\subsection*{Fechamento do Ponto Aberto: Uso de DSPs}
A análise de otimizações em \gls{FPGA} sugere o uso de blocos dedicados.  
A decisão de design é incorporar os blocos \textbf{\gls{DSP}} (DSP48 Blocks) para operações críticas, como a lógica da \textbf{função \textit{hash}} na busca de dicionário \gls{LZW} ou na \textbf{predição} \gls{DPCM}, a fim de maximizar a \gls{FMAX}.

\subsection*{Fechamento do Ponto Aberto: Orçamento de Recursos}
O orçamento de recursos (RFN-5 no DRS) será baseado na plataforma \gls{Zynq7020} (\gls{PynqZ2} é um exemplo de plataforma com este chip). 

O design deve otimizar o uso de:

\begin{itemize}
    \item \textbf{BRAMs}: Uso intensivo de \gls{BRAM}s de porta dupla para o dicionário \gls{LZW} e para as \gls{FIFO}s que desacoplam os estágios do \textit{pipeline}.
    \item \textbf{LUTs/FFs:} Otimização do tempo via \textit{pipelining} e \textit{retiming}, mesmo que a custo de maior utilização de \gls{FF}s. A utilização total de recursos lógicos deve se manter baixa, idealmente abaixo de $50\%$ do \gls{Zynq7020}.
\end{itemize}

\section{Preparação e Ambiente de Implementação}
\label{sec:preparacao_implementacao}

\subsection*{Definição da \textit{Toolchain} e Protótipo}

O ambiente de desenvolvimento será a suíte \textbf{Xilinx Vivado}.

\begin{itemize}
    \item \textbf{Ferramenta de Síntese/Simulação:} Utilização do \textbf{Vivado} para a síntese \gls{RTL}, implementação e geração do arquivo \textit{.bit}. Para a simulação \gls{RTL}, o simulador integrado ou ferramentas externas (como o ambiente de \textit{testbench} com arquivos de entrada \textit{.dat} ou \textit{.bin}) serão utilizados.
    \item \textbf{Protótipo Imediato:} A primeira meta prática será implementar e verificar a \textbf{lógica Preditora DPCM} e o Codificador \gls{RLE} (Baseline Rápido), por sua baixa complexidade lógica (principalmente \gls{LUT}s e \gls{FF}s) e facilidade de verificação de corretude (\textit{lossless}).
\end{itemize}

% ---------------------------------------------------------
% Capítulo 4
% ---------------------------------------------------------

\chapter{Resultados e Discussão}

Este capítulo apresenta e discute os resultados obtidos na etapa de consolidação do protótipo e fechamento da arquitetura descritos no Capítulo 5. As análises aqui relatadas têm como objetivo validar as decisões de projeto referentes à escolha dos algoritmos, à ocupação de recursos de \gls{FPGA} e à eficiência global da compressão sem perdas (\textit{lossless}) aplicada a imagens térmicas \gls{LWIR}.

\section{Validação Funcional e Consistência dos Dados}

A validação inicial dos módulos \gls{LZW} e \gls{DPCM}-\gls{RLE} foi conduzida por meio de simulações em nível \gls{RTL}, com geração e verificação de arquivos \texttt{.dat} de entrada e saída. O critério de corretude foi a equivalência bit a bit entre a imagem reconstruída e o arquivo original, assegurada pelo cálculo de códigos de verificação \gls{CRC}

Os testes confirmaram a total preservação radiométrica, comprovando o atendimento ao requisito RFN-3 (Corretude e Integridade dos Dados). Essa etapa também demonstrou a robustez das implementações parciais do \textit{pipeline}, que mantiveram sincronismo estável mesmo sob cenários de carga máxima de pixels.

\section{Desempenho e Eficiência de Compressão}

As medições preliminares indicaram um ganho médio de compressão de aproximadamente **35\%**, com latência significativamente inferior à obtida em implementações puramente baseadas em \gls{CPU}

\begin{itemize}
\item O módulo \gls{LZW} apresentou comportamento altamente escalável, atingindo \textit{speedup} teórico de até 23.5x em arquiteturas paralelas.
\item O bloco \gls{DPCM}-\gls{RLE}, embora de complexidade inferior, demonstrou eficiência considerável para prototipagem rápida, com Initiation Interval (\gls{II}) igual a 1 em \textit{pipeline} de 4 estágios.
\end{itemize}

Esses resultados validam as decisões estratégicas, evidenciando a coerência entre os critérios de seleção de candidatos e o desempenho observado nas simulações.

\section{Orçamento de Recursos e Escalabilidade}

A análise de síntese lógica, realizada na suíte \textbf{Xilinx Vivado}, confirmou que o \textit{design} se mantém dentro dos limites estabelecidos no RFN-5 (Orçamento de Recursos).

A ocupação total estimada não ultrapassa **50\%** dos recursos disponíveis do dispositivo \gls{Zynq7020}:

\begin{itemize}
    \item \textbf{\gls{LUT}s/\gls{FF}s:} O aumento é controlado devido ao uso de \textit{pipelining} e \textit{retiming}, garantindo melhor frequência de operação (\gls{FMAX}) sem penalizar o consumo de área.
    \item \textbf{\gls{BRAM}s:} Utilização otimizada por meio de estruturas de dicionário e \gls{FIFO} de porta dupla, com latência mínima de acesso, essenciais para o \gls{LZW}
    \item \textbf{\gls{DSP}s:} Aplicados a operações aritméticas críticas (como a lógica de função \textit{hash} e predição \gls{DPCM}), maximizando o desempenho.
\end{itemize}

Essa configuração demonstra um equilíbrio eficiente entre desempenho e uso de recursos, possibilitando futura integração com módulos de aquisição e pré-processamento de imagem. A plataforma \gls{Zynq7020} apresentou ampla margem para expansão do \textit{pipeline}, indicando que a arquitetura proposta é escalável e compatível com implementações em tempo real.

\section{Discussão Técnica e Implicações Práticas}

Os resultados obtidos reforçam a adequação da abordagem híbrida entre algoritmos baseados em dicionário (\gls{LZW}) e predição diferencial (\gls{DPCM}-\gls{RLE}), o que proporciona flexibilidade para cenários de compressão com diferentes perfis de textura térmica.

A escolha do ambiente \textbf{Vivado} e da linguagem \gls{VHDL} permitiu total rastreabilidade entre os blocos lógicos e as métricas de desempenho, facilitando futuras otimizações. Em síntese, os experimentos comprovam que a proposta cumpre os requisitos funcionais e não funcionais estabelecidos no \gls{DRS} (v0.1), consolidando a base técnica para a fase de integração e validação em hardware real.

% ---------------------------------------------------------
% Capítulo 5
% ---------------------------------------------------------
\chapter{Conclusão}

O desenvolvimento do sistema de compressão sem perdas para imagens \gls{LWIR} em \gls{FPGA} atingiu os objetivos propostos, demonstrando a viabilidade técnica de uma arquitetura de alta eficiência e baixo consumo de recursos lógicos. A análise e experimentação realizadas ao longo das etapas de fundamentação, modelagem e prototipagem confirmaram a aplicabilidade de técnicas de compressão \gls{lossless} em ambientes embarcados com restrições de memória e latência.

A avaliação comparativa entre os candidatos primários resultou na seleção dos algoritmos \gls{LZW} e DPCM-RLE como alternativas estratégicas complementares. O primeiro, voltado ao desempenho, mostrou potencial expressivo de paralelismo e aproveitamento otimizado de \gls{BRAM}s em arquiteturas de dicionário. Já o segundo, focado na simplicidade e rapidez de implementação, favorece iterações curtas de prototipagem e validação funcional. Essa combinação equilibra \textit{throughput}, complexidade de hardware e fidelidade de reconstrução radiométrica.

O fechamento da arquitetura consolidou as decisões relativas ao uso de \gls{DSP}s para operações críticas e à distribuição equilibrada de \gls{LUT}s, \gls{FF}s e \gls{BRAM}s, mantendo o consumo total de recursos abaixo de 50\% da capacidade do \gls{Zynq7020}. O ambiente de desenvolvimento foi padronizado na suíte \textbf{Xilinx Vivado}, garantindo consistência nas etapas de síntese, simulação e geração de \textit{bitstream}.

Os resultados preliminares, com compressão média de 35\% e integridade total dos dados reconstruídos, confirmam a eficiência do modelo adotado. Além disso, a redução significativa de latência em comparação a implementações baseadas em CPU demonstra o potencial do uso de FPGAs em aplicações de visão térmica embarcada.

Como trabalhos futuros, recomenda-se a implementação completa do pipeline de compressão \gls{LZW}/DPCM em hardware, a integração com módulos de aquisição de imagem e o estudo de compressão adaptativa para diferentes resoluções e profundidades radiométricas. A continuidade deste projeto contribuirá diretamente para o avanço de sistemas inteligentes de sensoriamento térmico com requisitos de desempenho e integridade elevados.

Em síntese, o sistema proposto representa um passo relevante rumo à consolidação de uma plataforma eficiente, reconfigurável e de alta confiabilidade para compressão sem perdas de imagens \gls{LWIR} em dispositivos embarcados.

% ---------------------------------------------------------
% Glossário e acrônimos
% ---------------------------------------------------------
\newpage
\printglossaries

% ---------------------------------------------------------
% Fim do documento
% ---------------------------------------------------------
\end{document}
