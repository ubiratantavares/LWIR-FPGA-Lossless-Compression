\documentclass[12pt,a4paper]{article}

% Pacotes básicos
\usepackage[utf8]{inputenc}
\usepackage[T1]{fontenc}
\usepackage[brazil]{babel}
\usepackage{graphicx}
\usepackage{geometry}
\usepackage{booktabs}
\usepackage{float}
\usepackage{hyperref}
\usepackage{listings}
\usepackage{xcolor}

% Configuração de margens
\geometry{margin=2.5cm}

% Configuração de código
\definecolor{codegreen}{rgb}{0,0.6,0}
\definecolor{codegray}{rgb}{0.5,0.5,0.5}
\definecolor{codepurple}{rgb}{0.58,0,0.82}
\definecolor{backcolour}{rgb}{0.95,0.95,0.92}

\lstdefinestyle{sdcstyle}{
    backgroundcolor=\color{backcolour},   
    commentstyle=\color{codegreen},
    keywordstyle=\color{magenta},
    numberstyle=\tiny\color{codegray},
    stringstyle=\color{codepurple},
    basicstyle=\ttfamily\footnotesize,
    breakatwhitespace=false,         
    breaklines=true,                 
    captionpos=b,                    
    keepspaces=true,                 
    numbers=left,                    
    numbersep=5pt,                  
    showspaces=false,                
    showstringspaces=false,
    showtabs=false,                  
    tabsize=2
}

\lstset{style=sdcstyle}

% Título e Autor
\title{\textbf{Relatório Parcial de Execução da Sprint 2}\\ \large Refinamento das Restrições de Timing (SDC)}
\author{Invent Vision - Projeto LWIR FPGA Lossless Compression}
\date{07 de Janeiro de 2026}

\begin{document}

\maketitle

\section{Introdução}
A \textbf{Sprint 2} foca na otimização do pipeline para atingir as metas de temporização ($F_{MAX} \ge 100 \text{ MHz}$) e throughput ($II=1$). Este relatório parcial documenta a atividade de refinamento das restrições de timing (Passo 2.1), essencial para guiar o processo de síntese e otimização do Quartus II.

\section{Refinamento das Restrições (Passo 2.1)}

O arquivo de restrições \texttt{timing.sdc} foi atualizado para refletir um cenário de operação mais realista e robusto.

\subsection{Definições de Clock}
Além da definição básica do período, foram adicionadas diretivas para calcular automaticamente as incertezas (*jitter* e *skew*) e derivar clocks gerados por PLLs (se utilizados futuramente).

\begin{lstlisting}[language=tcl, caption=Restrições de Clock]
create_clock -name clk -period 10.000 [get_ports {clk}]
derive_clock_uncertainty
derive_pll_clocks
\end{lstlisting}

\subsection{Restrições de I/O (Input/Output Delays)}
Para garantir que o design possa se comunicar corretamente com dispositivos externos (ex: sensor térmico ou memória), foram definidos atrasos de entrada e saída. Estes valores informam ao Quartus quanto tempo do ciclo de clock é consumido fora da FPGA, restringindo o orçamento de tempo interno.

\begin{itemize}
    \item \textbf{Input Delay:} 2.0 ns (máx). O dado chega aos pinos da FPGA até 2ns após a borda do clock externo.
    \item \textbf{Output Delay:} 2.0 ns (máx). O dispositivo externo precisa que o dado esteja estável 2ns antes da borda do clock.
\end{itemize}

Isso deixa efetivamente $10ns - 2ns - 2ns = 6ns$ para a lógica interna nos caminhos de I/O, forçando o sintetizador a otimizar essas interfaces.

\section{Conclusão Parcial}
Com as restrições refinadas, o ambiente de síntese está preparado para realizar uma análise de timing rigorosa (TimeQuest). As próximas atividades dependerão da execução da síntese para identificar e corrigir violações reais nos caminhos críticos.

\end{document}
